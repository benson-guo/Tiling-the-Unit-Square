\documentclass[11pt]{article}
\usepackage{latexsym}
\usepackage{amssymb}
\usepackage{amsfonts}
\usepackage{amsmath}
\usepackage{amsthm}
\usepackage[all]{xy}		%xy-pic
\usepackage{graphics}
%\usepackage{psfrag, psfig}
\usepackage[dvips]{graphicx}


%\swapnumbers
\theoremstyle{plain}%default
\newtheorem{thm}{Theorem}[section]
\newtheorem{lem}[thm]{Lemma}
\newtheorem{prop}[thm]{Proposition}
\newtheorem{cor}[thm]{Corollary}

\theoremstyle{definition}
\newtheorem{defn}[thm]{Definition}
\newtheorem{eg}{Example}
\newtheorem*{claim}{Claim}

\theoremstyle{remark}
\newtheorem{rmk}[thm]{Remark}
\newtheorem{notn}[thm]{Notation}

\numberwithin{equation}{section}


\title{On Tiling the Unit Square}
\author{Benson Guo and Peter Hazard}
\date{\today}

\begin{document}

\pagestyle{empty}
\noindent
Dear Benson,
\vskip 10pt
Here is a template for a Latex file to keep as a log of our project.
On the second page I have added a comments section, where we can put comments, rough work, calculations, questions, etc.
These should be added as a list, with the authors name (BG or PH resp.) and the date of the comment.
\vskip 50pt

{\flushright --PeH, \today.--}
%\flushleft
\newpage

\paragraph{Comments.}
\begin{itemize}
\item[PH -- 2015/02/01]
The sections ``Cross Sections'' and onwards are rough work, and will eventually (probably) be scrapped.
\end{itemize}
\begin{itemize}
\item[BG -- 2015/02/07]
We started the investigation of the viability of a greedy algorithm.
The algorithm works as follows:
\begin{enumerate}
\item Assume the longer side of a rectangle $R$ has a length $\frac{a}{b}$, and shorter side with a length $\frac{a\prime}{b\prime}$
\item Fill $R$ with a piece with a side length $\frac{1}{n_{0}}\times\frac{1}{n_{0}+1}  $ where $\frac{1}{n_0}$ satisfies $\frac{1}{n_0}\le\frac{a\prime}{b\prime}<\frac{1}{n_{0}-1}$
\item Cut the remaining area into 2 rectangles:
\begin{quote}
$R_0: \frac{1}{n+1}\times(\frac{a\prime}{b\prime}-\frac{1}{n_0})  $
\end{quote} 
\begin{quote}
$R_1: (\frac{a}{b}-\frac{1}{n+1})\times \frac{a\prime}{b\prime} $
\end{quote} 
\item Apply Step 1 to rectangle $R_0$.
\item Stop when $\frac{a\prime}{b\prime} - \sum \frac{1}{n_i}=0$
\item Apply Step 1 to next left-most rectangle.
\end{enumerate}
Studying  $R_0: \frac{1}{n_0+1}\times(\frac{a\prime}{b\prime}-\frac{1}{n_0})  $, we ask ourselves, is $\frac{a\prime}{b\prime}-\frac{1}{n_0}>\frac{1}{n_0+1}$?
By hypothesis, $\frac{1}{n_0}\le\frac{a\prime}{b\prime}<\frac{1}{n_{0}-1} \Rightarrow 0\le\frac{a\prime}{b\prime}-\frac{1}{n_0}<\frac{1}{n_0-1}-\frac{1}{n_0}$

$\frac{1}{n_0-1}-\frac{1}{n_0}=\frac{n_0-(n_0-1)}{n_0(n_0-1)}=\frac{1}{n_0(n_0-1)}$
Is $\frac{1}{n_0(n_0-1)}<\frac{1}{n_0+1}$?

$n_0+1<n_0^2-n_0\Rightarrow n_0^{2}-2n_0-1>0$

Hence, $\frac{a\prime}{b\prime}-\frac{1}{n_i}<\frac{1}{n_i(n_i-1)}<\frac{1}{n_i+1}$ for $n_i\ge2$.
This implicates that the shorter side will be on the vertical edges, which will approach $0$ as $\frac{1}{n_i}$ decreases. When it reaches $0$, the left side will be completely covered with smaller rectangles. 
\end{itemize}
\begin{itemize}
\item[BG -- 2015/02/16]
I coded a rough simulation of the greedy algorithm to answer the question of whether or not some rectangle will be required to be used twice. Very quickly I found a case in which the same rectangle is called upon consecutively. Take the case when you have a rectangle with dimensions $\frac{1}{n}\times\frac{1}{n+x}$. The algorithm will tile it with $\frac{1}{n+x}\times\frac{1}{n+x+1}$ and the remaining area will be, $\frac{1}{n}\times(\frac{1}{n}-\frac{1}{n+x+1})$. In this case $\frac{1}{n+x}\times\frac{1}{n+x+1}$ will be used again, provided $\frac{1}{n+x+1}$ is still the shorter side. Specifically, one case I came across was $\frac{1}{11}\times\frac{1}{140}$ which would be tiled with $\frac{1}{140}\times\frac{1}{141}$ leaving a rectangle of $\frac{130}{1551}\times\frac{1}{140}$. I will play around with the algorithm to see how we can go around this.
\item[PH -- 2015/02/18]
Observe that there are two choices for the greedy map. 
Namely we can choose either
\begin{equation}
G\colon\frac{x}{y}\mapsto \frac{x}{y}-\frac{1}{m}, \qquad \frac{1}{m}\leq \frac{x}{y}<\frac{1}{m-1}
\end{equation}
or
\begin{equation}
\tilde G\colon\frac{x}{y}\mapsto \frac{x}{y}-\frac{1}{m}, \qquad \frac{1}{m}< \frac{x}{y}\leq\frac{1}{m-1}
\end{equation}
(Notice the difference in domains.)
I think the first map was the one I introduced.
However, the second map also terminates after finitely many step.
In fact, in the `boundary' case of $\frac{x}{y}=\frac{1}{n}$ (the only case we really need to consider) we get
\begin{equation}
\frac{1}{n}-\frac{1}{n+1}=\frac{1}{n(n+1)}
\end{equation}
i.e. if the first map terminates after $k$ iterates, then the second terminates after $k+1$ iterates.

Therefore try implementing the algorithm using the second greedy map $\tilde G$ instead (or a mixture of the two).

I also started an appendix on properties of related maps (not-so-greedy maps). Currently, besides the definition of the family,  it only contains a question.
\end {itemize}

\begin{itemize}
\item[BG -- 2015/02/24]
A recap of what we discussed the previous weekend:

It was proved that no 2-tile filling exists (Rectangle composed of two tiles). From this, you can infer that no 3 or 4 tiling exists. Note that a 5 tiling arrangements can exist.

Definition: Let $R$ be a rectangle. Assume it is tiled with tiles $T_1, T_2,\ldots$

We say $R$ is reducible if there is a subrectangle $R\prime\subset R$ and tiles  $T_{i1}, T_{i2},\ldots$ such that $R\prime $ is tiled by $T_1, T_2,\ldots$ (and $R\prime$ doesn't consist of a single tile)

Claim: There exists an increasing sequence $N_1, N_2, \ldots$ such that for each $N_j$ there exists an irreducible tiling configuration with $j$ pieces.

Proof:
\begin{enumerate}
\item Step 1: Start with an irreducible rectangle $R$ that is a $N_j$ tiling.
\item Step 2: By induction, assume, $N_j$ is maximal.
\item Step 3: Take a piece $T$ touching the boundary of $R_j$.
\item Step 4: Extend $T$ outside the boundary, and let it be called $\tilde{T}$.
\item Step 5: Complete the new rectangle $\tilde{R}$ by adding rectanges alongside $\tilde{T}$, let the sections be called $S_1, S_2$
\item Step 6: Assume $\tilde{R}$ is reducible. Then $\exists  \tilde{R\prime}\subset \tilde{R}$ which is tiled.
\item Step 7: If $\tilde{R\prime}\subset R$ we get a contradiction ($R$ is irreducible by assumption).
\item Step 8: Hence, $\tilde{R\prime}$ contains at least one of $S_1, S_2, \tilde{T}$.


Case 1: $\tilde{R\prime}$ contains exactly one of $S_1, S_2, \tilde{T}$. 

-$S_i$, removing $S_i$ leaves $\tilde{R\prime \prime}=\tilde{R\prime}\setminus S_i$ tiled, which is a contradiction

-$\tilde{T}$, removing $\tilde{T}$ leaves $\tilde{R\prime \prime}=(\tilde{R\prime}\setminus \tilde{T})\cup T$, also a contradiction


Case 2: Case 1 fails in the case that $\tilde{R\prime \prime}$ is a single tile

\end {enumerate}

Questions:

Are "towers" being formed from the greedy algorithm/rotational algorithm?

Are the algorithms convergent? (Area of rectangle-tiles approaches 0)

How long do tiles remain unchanged in the rotational algorithm?

How can we prove the non-rotational algorithm always fails in a finite amount of steps?
\end {itemize}

\begin {itemize}
\item [BG -- 2015/03/08]
This week we looked at a new type of algorithm with a focus on the perimeter, and went back to the analysis of the old rotational algorithm.

This algorithm is based on two types of ways to place tiles:
\begin{enumerate}
\item 1.  If possible, and if the resulting free space is roughly proportional to a square, add a tile that reduces the perimeter.
\item 2. Otherwise, apply a splitting algorithm.
\end {enumerate}

During the tiling of approximately the $n$th tile, let the free spaces where a tile can be placed be $F_n^1,\ldots,F_n^{k_n}$

The double splitting algorithm places a tile in a free space that is roughly double the dimension of the tile. Hence, we can expect the following approximations to be preserved.

$F_N^{1} \sim \frac{2}{N+1} \times \frac {2}{N}$

$F_N^{1} \sim \frac{2}{N+2} \times \frac {2}{N+1}$

$\ldots$

$F_N^{N} \sim \frac{2}{N+N} \times \frac {2}{N+N-1}$

Note that a piece can only reduce the perimeter if it is placed in the first or last free space, and also has a length such that it completely fills part of the free space.


Questions to investigate for the upcoming weeks:

In the doubled splitting algorithm, does there exist a configuration of tiles $T_1, T_2, \ldots,T_N$ such that the free space satisifes the hypotheses, that the free spaces are approximately double the dimensions of unit tiles?

Is there any other proportion of free blocks that is better suited for tiling?

Does the perimeter of free space stop decreasing after some N in the roational algorithm?

Is there more than one limit point? Perhaps a limit arc that is inside $R$?

\end {itemize}

\begin{itemize}
\item [BG -- 2015/03/22]
This week we discussed a strategy for creating an infinite tiling of the unit square, based on the double-splitting algorithm, mentioned in the previous weeks. 

We discovered that to keep our assumption, of a corner tile being roughly in the proportion of $2:1$, where $F_N^{1} \sim \frac{1}{N_j} \times \frac {1}{N_j}$, we would have to tile it in the future on the tile in the order of $\frac{3N_j}{2}$

\end {itemize}


\newpage

\maketitle

\begin{abstract}
We consider the problem, originally due to L.\ Moser, of tiling the unit square with rectangles of side lengths $\frac{1}{n}\times\frac{1}{n+1}$ for $n=1,2,\ldots$.
This is the result of a University of Toronto Mentoring Program 2015.
\end{abstract}
\section{Introduction}
We consider the following:
\begin{quote}
{\it Problem:}
Can the unit square $[0,1]^2$ be tiled with rectangles of sides $\frac{1}{n}\times\frac{1}{n+1}$ for $n=1,2,\ldots$
\end{quote}
A more general formulation is the following:
\begin{quote}
{\it Problem:}
Find the smallest square $[0,1+\epsilon]^2$, $\epsilon\geq 0$, into which the rectangles of sides $\frac{1}{n}\times\frac{1}{n+1}$, $n=1,2,\ldots$, can be packed.
\end{quote}
According to~\cite[Section D11]{CFG94} these problems are originally due to L.\ Moser.
The best result for the second problem, to the authors knowledge, is due to Balint~\cite{Balint92, Balint98}. See also Paulhus~\cite{Paulhus98}.

\section{Preliminaries}
Let $T_0=[0,1]^2$ denote the unit square.
For each $n\in\mathbb{N}$, let $T_n$ denote the tile of side lengths $\frac{1}{n}\times\frac{1}{n+1}$.
In our considerations we shall make the following hypotheses about any tiling of $T_0$:
\begin{quote}
{\it Hypothesis 1:} 
The tiles are added parallel to the sides of $T_0$.
\end{quote}
\begin{quote}
{\it Hypothesis 2:}
For each $n\in\mathbb{N}$, the free space $T_0\setminus \bigcup_{n=1}^{n}T_i$ is connected and simply connected.
\end{quote}
Assume a tiling is given.
Let 
\begin{equation}
V_0=T_0, \qquad V_n=V_0\setminus \bigcup_{i=1}^{n}T_i, \quad i=1,2,\ldots
\end{equation}
Then, by the above hypotheses, for each positive integer $n$ the region $V_n$ is a polygon with sides aligned with the horizontal and vertical directions.
Let $V_n$ have vertices $v_n^1,\ldots,v_n^{r_n}$.
Consider the set of all horizontal and vertical lines through the $v_n^j$.
Take their complement in $V_n$.
This complement consists of finitely many rectangles $F_n^1,\ldots,F_n^{k_n}$ with sides parallel to the sides of $T_0$.
Furthermore, $V_n=\bigcup_{i=1}^{k_n} F_n^i$.

We make the following assumptions:
\begin{quote}
{\it Assumption 1:}
For each positive integer $n$, the tile $T_n$ is positioned in $V_{n-1}$ so that at least one vertex of $T_n$ lies at one of the vertices of $V_{n-1}$.
\end{quote}
\begin{quote}
{\it Assumption 2:}
For each positive integer $n$, if the tile $T_n$ fits into one of the free spaces $F_{n-1}^j$, then it must be positioned in one of them.
\end{quote}
Now we ask the following questions:
\begin{enumerate}
\item Does a tiling exist satisfying Assumptions 1 and 2?
\item Does such a tiling have the property that for each positive integer $n$, each $F_n^i$ except one, is eventually tiled by finitely many tiles?
\end{enumerate}
Since each $F_n^i$ has side-lengths of the form $\sum \frac{1}{n_i}-\sum \frac{1}{m_j}$ we therefore ask, firstly, if the sides can be `tiled' by a finite number of intervals of lengths $\frac{1}{l_1},\frac{1}{l_2},\ldots,\frac{1}{l_k}$, where $l_1,l_2,\ldots,l_k$ are distinct positive integers and $\min_r l_r >\max (n_i,m_j)$.
In the case when we do not have the second contraint we know the following, originally due to Fibonacci and rediscovered by Sylvester.
\begin{thm}
Any positive rational number $\frac{a}{b}$ can be expressed as a finite sum of unit fractions, i.e. there exist distinct positive integers $n_1,n_2,\ldots,n_k$ such that
\begin{equation}
\frac{a}{b}=\frac{1}{n_1}+\frac{1}{n_2}+\cdots+\frac{1}{n_k}
\end{equation}
\end{thm}


\subsection{Cross Sections}
Assume a tiling exists. 
Take a line parallel to the horizontal or vertical sides of $T_0$ and intersect it with $T_0$.
Assume that the tiles $T_{i_1},T_{i_2},\ldots$ intersect this line horizontally and the tiles $T_{j_1},T_{j_2}$ intersect it vertically.
Then
\begin{equation}
\sum\frac{1}{i_k}+\sum\frac{1}{j_k-1}=1
\end{equation}
The problem of finding positive integers $i_k, j_k$ satisfying this equation is classical. 
(See Klee and Wagon~\cite[Chapter 2.15]{KleeWagon91}.)

Assume there does exist a tiling of the unit square. 
Note that if we take a horizontal section, i.e. intersect with $\{y=y_0\}$ we find the length decomposes as
\begin{equation}
1=|l(S(y_0))|=\sum |l(S_{n_i(y_0)}(y_0))|=\sum\frac{1}{n_i(y_0)}
\end{equation}

\section{Perimeter Representation}
Observe that if a tiling exists then the perimeter of the free space must tend to zero, i.e. $\lim_{n\to \infty}|P(V_n)|=0$.

\subsection{Winding Numbers: How to tell adding a tile is admissible.}

\newpage
\begin{appendix}
\section{The Not-So-Greedy Algorithm}
Consider the following map
\begin{equation}
\frac{a}{b}\mapsto \frac{a}{b}-\frac{1}{m+1}, \qquad \mbox{where} \qquad \frac{1}{m}\leq \frac{a}{b}<\frac{1}{m-1}
\end{equation}
More generally we can consider the $k$-th greedy map given by
\begin{equation}
G_k(x)=x-\frac{1}{\lceil x\rceil +k}
\end{equation}
The above gives $G_1$ and $G_0$ gives the standard greedy map, extended to the set of real numbers.
As has been mentioned in the main text, the map $G_0$ terminates after finitely many iterates for each rational number.
We ask the following question.
\begin{quote}
{\it Question: }
For which real numbers does $G_1$ terminate after finitely many iterates?
For which real numbers does $G_k$ terminate after finitely many iterates?
\end{quote}
It is clear that this set is a subset of the set of rational numbers. Hence we ask for which rational numbers does this ``not-so-greedy' algorithm terminate?
\end{appendix}

\small
\addcontentsline{toc}{section}{Bibliography}
\bibliographystyle{amsplain}
\bibliography{BG-PH_Tiling02}


\end{document}
